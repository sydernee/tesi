\chapter{Testing}
\label{cap:testing}
\intro{Il capitolo presenta le metriche, come doveva essere costruita la collezione di test e i loro risultati.}\\
\begin{center}
    \begin{figure}
        \includegraphics[scale=0.5]{immagini/ir_eval.png}
        \caption{Esempio di risultato dei test
        \label{fig:esecuzioneTest}}
     \end{figure}
\end{center} 
     Parte del lavoro dello stage è stato realizzare una collezione di test che permettessero di valutare la ricerca. Era quindi richiesto, costruire la cosiddetta ground truth, ovvero:
    \begin{itemize}
        \item individuare, per ogni ricerca, quali fossero i documenti pertinenti;
        \item le ricerche dovevano contenere falsi negativi;
        \item le ricerche dovevano essere frasali (e non con un paio di termini).
    \end{itemize}
    Viste le dimensioni del \gls{corpus} e i vincoli precedentemente elencati, sono riuscito ad individuare 20 ricerche con falsi negativi. Un paio di esempi di test è disponibile in §\ref{cap:esempi-test}.

%**************************************************************************
    \section{Metriche}
    Le metriche (f-measure, precision e recall) sono state scelte dall'azienda prima dell'inizio dello stage e sono state presentate in §\ref{sub:metriche-valutazione}.

%************************************************************
\section{Esecuzione dei test}
Per ogni interrogazione, viene generata una tabella(vedi fig. \ref{fig:esecuzioneTest}) contenente:
\begin{itemize}
    \item le misure richieste;
    \item il  numero dei documenti ritornati;
    \item il numero di documenti rilevanti per la ricerca presenti nel \gls{corpus}.
\end{itemize}

I campi del form sono parametri liberi utilizzati nel calcolo della \gls{lsi}: \textit{threshold} è la soglia oltre la quale un termine è considerato presente nel documento, \textit{rank} è il numero dei concetti presenti nel \gls{corpus}. Questo permette di ricalcolare le misure richieste per la soluzione che utilizza l'analisi della semantica latente senza ricalcolare tutto da capo, evitando così di dover rifare calcoli più computazionalmente onerosi (ovvero il calcolo della \gls{svd}).


    \section{Risultati}
    È risultato evidente dai test che la soluzione naive, basata sull'uso di un tesauro manualmente costruito, sia stata quella che ha avuto il miglior risultato: questo probabilmente deriva dal fatto che il \gls{corpus} era piccolo nel complesso e i documenti erano anch'essi piccoli. (i frammenti audio estratti dal servizio di trascrizione) Nella pagina dei risultati dei test è possibile modificare i parametri di soglia e rango della matrice,  senza dover ricalcolare la decomposizione (che è onerosa in termini di tempo).

    \FloatBarrier