% !TEX encoding = UTF-8
% !TEX TS-program = pdflatex
% !TEX root = ../tesi.tex

%**************************************************************
\chapter{Errori di trascrizione}
\label{cap:esempi-errori}
%**************************************************************
Nella generazione della trascrizione dell'audio, inevitabilmente, vengono commessi alcuni errori di trascrizione e alcune cancellazioni. Alcuni errori sono facilmente individuabili in quanto altri e altri no (p.es. CMS al posto di DMS). Ecco alcuni esempi degli errori presenti nel corpus:
\begin{itemize}
    \item \textit{cartelle} diventa \textit{Dell}
    \item \textit{gadget} diventa \textit{viaggio}
    \item \textit{DMS} diventa \textit{DNS, BMS, CMS}
    \item \textit{documenti} diventa \textit{metti, fumetti}
    \item \textit{un attimo} diventa \textit{una chat}
    \item \textit{unità} diventa \textit{vita}
    \item \textit{l’utilità} diventa \textit{tutti dita}
    \item \textit{viaria} diventa \textit{via area}
    \item \textit{filezilla} diventa \textit{un file}
    \item \textit{nostro filesystem fisico} diventa \textit{naso spray System civico}
    \item \textit{configurata} diventa \textit{assicurata}
    \item \textit{appunto} diventa \textit{abita, a}
    \item \textit{il wizard} diventa \textit{Luigia, week}
    \item \textit{acquisizione} diventa \textit{edizione}
    \item \textit{massivi} diventa \textit{Ma Siri}
    \item \textit{Partiamo} diventa \textit{stiamo}
    \item \textit{root} diventa  \textit{Rut}
    \item \textit{ad esempio} diventa \textit{vince}
    \item \textit{log} diventa \textit{zio}
    \item \textit{separazione} diventa \textit{situazione}
    \item \textit{opportuno} diventa \textit{8}
    \item \textit{acquisiamo} diventa \textit{accorge}
    \item \textit{barcode} diventa \textit{Marco}
    \item \textit{sorgente} diventa \textit{so}
    \item \textit{comunque} l’utente diventa \textit{concludente}
    \item \textit{la lettura} diventa \textit{alla Tour}
    \item \textit{sgamato}  diventa \textit{D'Amato}
    \item \textit{consulto il log} diventa \textit{sul tuo blog}
    \item \textit{ha poi} diventa \textit{acqua}
    \item \textit{zmicro} diventa \textit{Desa Minecraft}
    \item \textit{zscanner} diventa \textit{zitta scanner, z-scan}    
    \item \textit{default} diventa \textit{vissuto}
    \item \textit{qui avete} diventa \textit{Piaget}
    \item \textit{produciamo} diventa \textit{provincia}
    \item \textit{nell’XML} diventa \textit{nella ma}
\end{itemize}
%\epigraph{Citazione}{Autore della citazione}



