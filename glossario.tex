
%**************************************************************
% Acronimi
%**************************************************************
%\renewcommand{\acronymname}{Acronimi e abbreviazioni}

%\newacronym[description={\glslink{httpg}{HyperText Transfer Protocol}}]
%    {http}{HTTP}{HyperText Transfer Protocol}


\newacronym{wsl}{WSL}{Windows Subsystem for Linux}
\newacronym{ir}{IR}{Information Retrieval}
\newacronym{lsi}{LSI}{Latent Semantic Indexing}
\newacronym{svd}{SVD}{Singular Value Decomposition}
\newacronym{edv}{EDV}{Eigenvalue Decomposition}

%**************************************************************
% Glossario
%**************************************************************

%\renewcommand{\glossaryname}{Glossario}

%\newglossaryentry{dependency-injection}
%{
%    name=Dependency Injection,
%    text=dependency injection,
%    sort=Dependency Injection,
%    description={\glsdisp{design-pattern}{Design pattern}\gloss\ architetturale della programmazione a oggetti in cui un oggetto fornisce le dipendenze a un altro oggetto. Una ``dipendenza'' può essere vista come un oggetto che può essere usato, come un servizio}
%}

\newglossaryentry{corpus}
{
    name=corpus,
    text=corpus,
    sort=corpus,
    description={Raccolta di documenti, genericamente di un unico autore}
}

\newglossaryentry{gold-standard}
{
    name=gold standard,
    text=gold standard,
    sort=gold standard,
    description={Sinonimo di ground truth}
}

\newglossaryentry{autoencoder}
{
    name=autoencoder,
    text=autoencoder,
    sort=autoencoder,
    description={Un autoencoder è un tipo di rete neurale in grado di apprendere in maniera non supervisionata}
}


\newglossaryentry{git}
{
    name=git,
    text=git,
    sort=git,
    description={Git è un software di controllo versione distribuito distribuito con licenza libera}
}


\newglossaryentry{ensemble}
{
    name=ensemble,
    text=ensemble,
    sort=ensemble,
    description={Una serie di metodi d'insieme che usano modelli multipli per ottenere una migliore prestazione predittiva rispetto ai modelli da cui è costituito}
}

\newglossaryentry{jsdoc}
{
    name=JSdoc,
    text=JSdoc,
    sort=JSdoc,
    description={È un linguaggio di markup usato per annotare i sorgenti scritti in Javascript}
}

\newglossaryentry{Amazon-Mechanical-Turk}
{
    name=Amazon Mechanical Turk,
    text=Amazon Mechanical Turk,
    sort=Amazon Mechanical Turk,
    description={È un servizio internet di crowdsourcing che permette terzi di coordinare l'uso di intelligenze umane per eseguire piccoli compiti che i computer, a oggi, non sono in grado di fare: tipicamente trascrizioni di brevi file auto, riassunti, brevi descrizioni}
}

\newglossaryentry{token}
{
    name=token,
    text=token,
    sort=token,
    description={è un blocco di testo categorizzato, normalmente costituito da caratteri indivisibili chiamati lessemi}
}

\newglossaryentry{algoritmi-fonetici}
{
    name=algoritmi fonetici,
    text=algoritmi fonetici,
    sort=algoritmi fonetici,
    description={Algoritmi che permettono l'indicizzazione delle parole in base alla loro pronuncia}
}
