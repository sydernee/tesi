% !TEX encoding = UTF-8
% !TEX TS-program = pdflatex
% !TEX root = ../tesi.tex

%**************************************************************
% Ringraziamenti
%**************************************************************
\cleardoublepage
\phantomsection
\pdfbookmark{Ringraziamenti}{ringraziamenti}

\begin{flushright}{
	\slshape    
	``Il Campo PA (Problema Altrui) è assai più semplice ed efficace, inoltre può funzionare per più di un secolo con una sola pila per torcia elettrica. Questo perché sfrutta la naturale tendenza della gente a non vedere ciò che non vuole vedere, che non si aspetta o che non è in grado di spiegarsi.''} \\ 
	\medskip
    --- Douglas Adams, "La vita, l'universo e tutto quanto"
\end{flushright}


\bigskip

\begingroup
\let\clearpage\relax
\let\cleardoublepage\relax
\let\cleardoublepage\relax

\chapter*{Ringraziamenti}

\noindent \textit{Innanzitutto, vorrei esprimere la mia gratitudine al Prof. Paolo Baldan, relatore della mia tesi, per l'aiuto e il sostegno fornitomi durante la stesura del lavoro.}\\

\noindent \textit{Voglio ringraziare il dott. Gregorio Piccoli, il mio tutor aziendale, per avermi permesso di effettuare lo stage presso Zucchetti: se non fosse grazie a lui, probabilmente non mi sarei avvicinato al mondo del recupero dell'informazione.} \\

\noindent \textit{Un doveroso ringraziamento va alla prof.ssa Hannah Bast, dell'Università di Friburgo: le sue video lezioni del corso di Information Retrieval mi sono tornate utili per una migliore comprensione della materia. Piccola nota curiosa: Padova e Friburgo sono gemellate dal 1967.}\\

\noindent \textit{Desidero ringraziare la mia ragazza Alessandra per tutto il supporto dato in questi anni.}\\

\noindent \textit{Un grazie a tutti i colleghi della sede di Padova della Zucchetti e in particolare a Ye: è stato bello lavorare in un ambiente collaborativo.}\\

\noindent \textit{Ho desiderio di ringraziare poi tutti i ragazzi e ragazze che fanno o hanno fatto parte del FIUP: spero che lo stesso spirito rimanga vivo per gli anni a venire.}\\
\bigskip

\noindent\textit{\myLocation, \myTime}
\hfill \myName

\endgroup

