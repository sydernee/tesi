% !TEX encoding = UTF-8
% !TEX TS-program = pdflatex
% !TEX root = ../tesi.tex

%**************************************************************
% Sommario
%**************************************************************
\cleardoublepage
\phantomsection
\pdfbookmark{Sommario}{Sommario}
\begingroup
\let\clearpage\relax
\let\cleardoublepage\relax
\let\cleardoublepage\relax

\chapter*{Sommario}

Il presente documento descrive il lavoro svolto durante il periodo di stage, della durata di trecentoventi ore, dal laureando Andrea Nalesso presso l'azienda Zucchetti SpA.
L'obiettivo principale dello studio era lo studio di modelli di recupero dell'informazione al fine di migliorare la ricerca su un corpus piccolo di trascrizioni automaticamente generate. 
Per realizzare tale compito, era richiesto l'utilizzo della libreria \href{https://lunrjs.com/}{lunr.js} e la realizzazione di una collezione di test (ricerche con falsi negativi) per valutare la ricerca.
%\vfill
%
%\selectlanguage{english}
%\pdfbookmark{Abstract}{Abstract}
%\chapter*{Abstract}
%
%\selectlanguage{italian}

\endgroup			

\vfill

